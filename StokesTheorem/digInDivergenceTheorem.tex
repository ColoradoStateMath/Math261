\documentclass{ximera}

\input{../preamble.tex}
\usepackage{tikz-cd}

%%\outcome{Understand Divergence Theorem and how it can be used.}

\title[Dig-In:]{Divergence Theorem}

\begin{document}
\begin{abstract}
%%% Nothing for now
\end{abstract}
\maketitle

%%%% Dan's Bates' Notes Dec 04 2015
For flux across the boundary of a solid.

\begin{definition}
The \underline{Divergence} of a vector field $\overrightarrow{F} = <M, N, P>$ is

\[
\text{div }\overrightarrow{F} = \nabla \cdot \overrightarrow{F} = <\frac{\partial}{\partial x}, \frac{\partial}{\partial y}, \frac{\partial}{\partial z}> \cdot <M,N,P> = M_x + N_y + P_z. 
\]
\end{definition}

\begin{theorem}[Divergence Theorem]
\[
\text{ Flux } = \underbrace{\iint_S \overrightarrow{F} \cdot \overrightarrow{n} \d \sigma}_{Surface Intergral}
= \underbrace{\iiint_D \nabla\cdot \overrightarrow{F} \d V}_{triple inegral}
\]
where the surface $S$ is the boundary of the solid $D$.
\end{theorem}

\begin{example}
Given a vector field $\overrightarrow{F} = <x,y,z>$ and surface $S: x^2 + y^2 + z^2 = a^2$ (a sphere of radius $a$), find the flux across $S$ in both ways.

\begin{explanation}
(\underline{Version 1: The surface integral:} $\text{ Flux } = \iint_S \overrightarrow{F} \cdot \overrightarrow{n} \d \sigma$)

We first need to compute $\overrightarrow{n}$, and then $\overrightarrow{n}\cdot \overrightarrow{F}$

    \begin{align*}
     \overrightarrow{n} &= \frac{\nabla f}{|\nabla f|} \\
    &= \frac{<2x, 2y, 2z>}{|<2x,2y,2z>|} \\
    &= \frac{2<x,y,z>}{\sqrt{4x^2 + 4y^2 +4z^2}} \\
    &= \frac{<x,y,z>}{\sqrt{x^2 + y^2 + z^2}} \\
    &= <\answer[given]{x/a}, \answer[given]{y/a},     \answer[given]{z/a}>
    \end{align*}

so... 

$\overrightarrow{F} \cdot \overrightarrow{n} = \answer[given]{a}$.

Therefore 
    \begin{align*}
     \text{Flux across }S &= \iint_S \overrightarrow{F} \cdot \overrightarrow{n} \d \sigma \\
    &= \iint_S a \d \sigma \\
    &= a \underbrace{\iint_S \d \sigma}_{surface of sphere} \\
    &= a(\answer[given]{4 \pi a^2}) \\
    &= \boxed{\answer[given]{4 \pi a^3}}.
    \end{align*}

\end{explanation}

\begin{explanation}
(\underline{Version 2: The triple integral}: Flux $=  \iiint_D \nabla \cdot \overrightarrow{F} \d V$)

We first need $\nabla \cdot \overrightarrow{F}$.

    \begin{align*}
     \nabla \cdot  \overrightarrow{F} &= \nabla \cdot <x,y,z> \\
    &= \answer[given]{3}.
    \end{align*}

So to compute Flux:

    \begin{align*}
     \text{Flux across }S &= \iint_S \overrightarrow{F} \cdot \overrightarrow{n} \d \sigma \\
    &= \iiint_D 3 \d V \\
    &= 3 \underbrace{\iiint_D \d V}_{volume of sphere} \\
    &= 3(\answer[given]{\frac{4}{3} \pi a^3}) \\
    &= \boxed{4 \pi a^3}.
    \end{align*}
\end{explanation}
\end{example}

Therefore we have two ways to compute the same  quantity. We can choose which ways is easier and do that!

\begin{example}
Find the flux across the boundary of the unit cube of the vector field $\overrightarrow{F} = <xy,yz,xz>$.

{\bf Insert picture of the unit cube here!}

\begin{explanation}
By the divergence theorem, we can compute this in two ways:
\underline{Option 1}: $\iint_S \overrightarrow{F} \cdot \overrightarrow{n} \d \sigma$
or \underline{Option 2}: $\iiint_D \nabla\cdot \overrightarrow{F} \d V$

Just looking at the landscape, Option 1 would be doing a surface integral for 6 different faces.

Option 2 on the other hand can be done in one fell swoop.

    \begin{align*}
     \iiint_D \nabla \cdot  \overrightarrow{F} &= \int_{\answer[given]{0}}^{\answer[given]{1}} \int_{\answer[given]{0}}^{\answer[given]{1}} \int_{\answer[given]{0}}^{\answer[given]{1}} (\answer[given]{x+y+z}) \d x \d y \d z \\
     &= \int_{\answer[given]{0}}^{\answer[given]{1}} \int_{\answer[given]{0}}^{\answer[given]{1}} (\answer[given]{y + z + 0.5}) \d y \d z \\
     &= \int_{\answer[given]{0}}^{\answer[given]{1}} (\answer[given]{z + 1})\d z\\
     &= \answer[given]{3/2}.
    \end{align*}
\end{explanation}


\end{example}




%\begin{explanation}
%    \begin{align*}
%      \text{curl} &= \nabla \times \overrightarrow{F} \\
%      &= <\frac{\partial}{\partial x}, \frac{\partial}{\partial y}, \frac{\partial}{\partial z}> \times <M, N, P> \\
%      &= <P_y - N_z, -(P_x - M_z), N_x - M_y>.\\
%      &= \overrightarrow{0} (\text{by the component test}).
%    \end{align*}
%\end{explanation}




\end{document}
