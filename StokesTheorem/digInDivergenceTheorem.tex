\documentclass{ximera}

%\usepackage{todonotes}

\newcommand{\todo}{}

\usepackage{tkz-euclide}
\tikzset{>=stealth} %% cool arrow head
\tikzset{shorten <>/.style={ shorten >=#1, shorten <=#1 } } %% allows shorter vectors

\usetikzlibrary{backgrounds} %% for boxes around graphs
\usetikzlibrary{shapes,positioning}  %% Clouds and stars
\usetikzlibrary{matrix} %% for matrix
\usepgfplotslibrary{polar} %% for polar plots
\usetkzobj{all}
\usepackage[makeroom]{cancel} %% for strike outs
%\usepackage{mathtools} %% for pretty underbrace % Breaks Ximera
\usepackage{multicol}





\usepackage{array}
\setlength{\extrarowheight}{+.1cm}   
\newdimen\digitwidth
\settowidth\digitwidth{9}
\def\divrule#1#2{
\noalign{\moveright#1\digitwidth
\vbox{\hrule width#2\digitwidth}}}





\newcommand{\RR}{\mathbb R}
\newcommand{\R}{\mathbb R}
\newcommand{\N}{\mathbb N}
\newcommand{\Z}{\mathbb Z}

%\renewcommand{\d}{\,d\!}
\renewcommand{\d}{\mathop{}\!d}
\newcommand{\dd}[2][]{\frac{\d #1}{\d #2}}
\newcommand{\pp}[2][]{\frac{\partial #1}{\partial #2}}
\renewcommand{\l}{\ell}
\newcommand{\ddx}{\frac{d}{\d x}}

\newcommand{\zeroOverZero}{\ensuremath{\boldsymbol{\tfrac{0}{0}}}}
\newcommand{\inftyOverInfty}{\ensuremath{\boldsymbol{\tfrac{\infty}{\infty}}}}
\newcommand{\zeroOverInfty}{\ensuremath{\boldsymbol{\tfrac{0}{\infty}}}}
\newcommand{\zeroTimesInfty}{\ensuremath{\small\boldsymbol{0\cdot \infty}}}
\newcommand{\inftyMinusInfty}{\ensuremath{\small\boldsymbol{\infty - \infty}}}
\newcommand{\oneToInfty}{\ensuremath{\boldsymbol{1^\infty}}}
\newcommand{\zeroToZero}{\ensuremath{\boldsymbol{0^0}}}
\newcommand{\inftyToZero}{\ensuremath{\boldsymbol{\infty^0}}}



\newcommand{\numOverZero}{\ensuremath{\boldsymbol{\tfrac{\#}{0}}}}
\newcommand{\dfn}{\textbf}
%\newcommand{\unit}{\,\mathrm}
\newcommand{\unit}{\mathop{}\!\mathrm}
\newcommand{\eval}[1]{\bigg[ #1 \bigg]}
\newcommand{\seq}[1]{\left( #1 \right)}
\renewcommand{\epsilon}{\varepsilon}
\renewcommand{\iff}{\Leftrightarrow}

\DeclareMathOperator{\arccot}{arccot}
\DeclareMathOperator{\arcsec}{arcsec}
\DeclareMathOperator{\arccsc}{arccsc}
\DeclareMathOperator{\si}{Si}
\DeclareMathOperator{\proj}{proj}
\DeclareMathOperator{\scal}{scal}


\newcommand{\tightoverset}[2]{% for arrow vec
  \mathop{#2}\limits^{\vbox to -.5ex{\kern-0.75ex\hbox{$#1$}\vss}}}
\newcommand{\arrowvec}[1]{\tightoverset{\scriptstyle\rightharpoonup}{#1}}
\renewcommand{\vec}{\mathbf}
\newcommand{\veci}{\vec{i}}
\newcommand{\vecj}{\vec{j}}
\newcommand{\veck}{\vec{k}}
\newcommand{\vecl}{\boldsymbol{\l}}

\newcommand{\dotp}{\bullet}
\newcommand{\cross}{\boldsymbol\times}
\newcommand{\grad}{\boldsymbol\nabla}
\newcommand{\divergence}{\grad\dotp}
\newcommand{\curl}{\grad\cross}
%\DeclareMathOperator{\divergence}{divergence}
%\DeclareMathOperator{\curl}[1]{\grad\cross #1}


\colorlet{textColor}{black} 
\colorlet{background}{white}
\colorlet{penColor}{blue!50!black} % Color of a curve in a plot
\colorlet{penColor2}{red!50!black}% Color of a curve in a plot
\colorlet{penColor3}{red!50!blue} % Color of a curve in a plot
\colorlet{penColor4}{green!50!black} % Color of a curve in a plot
\colorlet{penColor5}{orange!80!black} % Color of a curve in a plot
\colorlet{fill1}{penColor!20} % Color of fill in a plot
\colorlet{fill2}{penColor2!20} % Color of fill in a plot
\colorlet{fillp}{fill1} % Color of positive area
\colorlet{filln}{penColor2!20} % Color of negative area
\colorlet{fill3}{penColor3!20} % Fill
\colorlet{fill4}{penColor4!20} % Fill
\colorlet{fill5}{penColor5!20} % Fill
\colorlet{gridColor}{gray!50} % Color of grid in a plot

\newcommand{\surfaceColor}{violet}
\newcommand{\surfaceColorTwo}{redyellow}
\newcommand{\sliceColor}{greenyellow}




\pgfmathdeclarefunction{gauss}{2}{% gives gaussian
  \pgfmathparse{1/(#2*sqrt(2*pi))*exp(-((x-#1)^2)/(2*#2^2))}%
}


%%%%%%%%%%%%%
%% Vectors
%%%%%%%%%%%%%

%% Simple horiz vectors
\renewcommand{\vector}[1]{\left\langle #1\right\rangle}


%% %% Complex Horiz Vectors with angle brackets
%% \makeatletter
%% \renewcommand{\vector}[2][ , ]{\left\langle%
%%   \def\nextitem{\def\nextitem{#1}}%
%%   \@for \el:=#2\do{\nextitem\el}\right\rangle%
%% }
%% \makeatother

%% %% Vertical Vectors
%% \def\vector#1{\begin{bmatrix}\vecListA#1,,\end{bmatrix}}
%% \def\vecListA#1,{\if,#1,\else #1\cr \expandafter \vecListA \fi}

%%%%%%%%%%%%%
%% End of vectors
%%%%%%%%%%%%%

%\newcommand{\fullwidth}{}
%\newcommand{\normalwidth}{}



%% makes a snazzy t-chart for evaluating functions
%\newenvironment{tchart}{\rowcolors{2}{}{background!90!textColor}\array}{\endarray}

%%This is to help with formatting on future title pages.
\newenvironment{sectionOutcomes}{}{} 



%% Flowchart stuff
%\tikzstyle{startstop} = [rectangle, rounded corners, minimum width=3cm, minimum height=1cm,text centered, draw=black]
%\tikzstyle{question} = [rectangle, minimum width=3cm, minimum height=1cm, text centered, draw=black]
%\tikzstyle{decision} = [trapezium, trapezium left angle=70, trapezium right angle=110, minimum width=3cm, minimum height=1cm, text centered, draw=black]
%\tikzstyle{question} = [rectangle, rounded corners, minimum width=3cm, minimum height=1cm,text centered, draw=black]
%\tikzstyle{process} = [rectangle, minimum width=3cm, minimum height=1cm, text centered, draw=black]
%\tikzstyle{decision} = [trapezium, trapezium left angle=70, trapezium right angle=110, minimum width=3cm, minimum height=1cm, text centered, draw=black]

\usepackage{tikz-cd}

%%\outcome{Understand Divergence Theorem and how it can be used.}

\title[Dig-In:]{Divergence Theorem}

\begin{document}
\begin{abstract}
%%% Nothing for now
\end{abstract}
\maketitle

%%% From Dan Bates' Notes Dec 04
The \underline{Divergence} of a vector field $\overrightarrow{F} = <M, N, P>$ is

\begin{align*}
\text{div }\overrightarrow{F} &= \nabla \cdot \overrightarrow{F} \\
&= <\frac{\partial}{\partial x}, \frac{\partial}{\partial y}, \frac{\partial}{\partial z}> \cdot <M,N,P> \\
&= M_x + N_y + P_z.
\end{align*}


\begin{theorem}[Divergence Theorem]
\[
\text{ Flux } = \underbrace{\iint_S \overrightarrow{F} \cdot \overrightarrow{n} \d \sigma}_{Surface Intergral}
= \underbrace{\iiint_D \nabla\cdot \overrightarrow{F} \d V}_{triple inegral}
\]
where the surface $S$ is the boundary of the solid $D$.
\end{theorem}

\begin{example}
Given a vector field $\overrightarrow{F} = <x,y,z>$ and surface $S: x^2 + y^2 + z^2 = a^2$ (a sphere of radius $a$), find the flux across $S$ in both ways.

\begin{explanation}
(\underline{Version 1: The surface integral:} $\text{ Flux } = \iint_S \overrightarrow{F} \cdot \overrightarrow{n} \d \sigma$)

We first need to compute $\overrightarrow{n}$, and then $\overrightarrow{n}\cdot \overrightarrow{F}$

    \begin{align*}
     \overrightarrow{n} &= \frac{\nabla f}{|\nabla f|} \\
    &= \frac{<2x, 2y, 2z>}{|<2x,2y,2z>|} \\
    &= \frac{2<x,y,z>}{\sqrt{4x^2 + 4y^2 +4z^2}} \\
    &= \frac{<x,y,z>}{\sqrt{x^2 + y^2 + z^2}} \\
    &= <\answer[given]{x/a}, \answer[given]{y/a},     \answer[given]{z/a}>
    \end{align*}

so... 

$\overrightarrow{F} \cdot \overrightarrow{n} = \answer[given]{a}$.

Therefore 
    \begin{align*}
     \text{Flux across }S &= \iint_S \overrightarrow{F} \cdot \overrightarrow{n} \d \sigma \\
    &= \iint_S a \d \sigma \\
    &= a \underbrace{\iint_S \d \sigma}_{surface of sphere} \\
    &= a(\answer[given]{4 \pi a^2}) \\
    &= \boxed{\answer[given]{4 \pi a^3}}.
    \end{align*}

\end{explanation}

\begin{explanation}
(\underline{Version 2: The triple integral}: Flux $=  \iiint_D \nabla \cdot \overrightarrow{F} \d V$)

We first need $\nabla \cdot \overrightarrow{F}$.

    \begin{align*}
     \nabla \cdot  \overrightarrow{F} &= \nabla \cdot <x,y,z> \\
    &= \answer[given]{3}.
    \end{align*}

So to compute Flux:

    \begin{align*}
     \text{Flux across }S &= \iint_S \overrightarrow{F} \cdot \overrightarrow{n} \d \sigma \\
    &= \iiint_D 3 \d V \\
    &= 3 \underbrace{\iiint_D \d V}_{volume of sphere} \\
    &= 3(\answer[given]{\frac{4}{3} \pi a^3}) \\
    &= \boxed{4 \pi a^3}.
    \end{align*}
\end{explanation}
\end{example}

Therefore we have two ways to compute the same  quantity. We can choose which ways is easier and do that!

\begin{example}
Find the flux across the boundary of the unit cube of the vector field $\overrightarrow{F} = <xy,yz,xz>$.

{\bf Insert picture of the unit cube here!}

\begin{explanation}
By the divergence theorem, we can compute this in two ways:
\underline{Option 1}: $\iint_S \overrightarrow{F} \cdot \overrightarrow{n} \d \sigma$
or \underline{Option 2}: $\iiint_D \nabla\cdot \overrightarrow{F} \d V$

Just looking at the landscape, Option 1 would be doing a surface integral for 6 different faces.

Option 2 on the other hand can be done in one fell swoop.

    \begin{align*}
     \iiint_D \nabla \cdot  \overrightarrow{F} &= \int_{\answer[given]{0}}^{\answer[given]{1}} \int_{\answer[given]{0}}^{\answer[given]{1}} \int_{\answer[given]{0}}^{\answer[given]{1}} (\answer[given]{x+y+z}) \d x \d y \d z \\
     &= \int_{\answer[given]{0}}^{\answer[given]{1}} \int_{\answer[given]{0}}^{\answer[given]{1}} (\answer[given]{y + z + 0.5}) \d y \d z \\
     &= \int_{\answer[given]{0}}^{\answer[given]{1}} (\answer[given]{z + 1})\d z\\
     &= \answer[given]{3/2}.
    \end{align*}
\end{explanation}


\end{example}




%\begin{explanation}
%    \begin{align*}
%      \text{curl} &= \nabla \times \overrightarrow{F} \\
%      &= <\frac{\partial}{\partial x}, \frac{\partial}{\partial y}, \frac{\partial}{\partial z}> \times <M, N, P> \\
%      &= <P_y - N_z, -(P_x - M_z), N_x - M_y>.\\
%      &= \overrightarrow{0} (\text{by the component test}).
%    \end{align*}
%\end{explanation}




\end{document}
