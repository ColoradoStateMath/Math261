\documentclass{ximera}

%\usepackage{todonotes}

\newcommand{\todo}{}

\usepackage{tkz-euclide}
\tikzset{>=stealth} %% cool arrow head
\tikzset{shorten <>/.style={ shorten >=#1, shorten <=#1 } } %% allows shorter vectors

\usetikzlibrary{backgrounds} %% for boxes around graphs
\usetikzlibrary{shapes,positioning}  %% Clouds and stars
\usetikzlibrary{matrix} %% for matrix
\usepgfplotslibrary{polar} %% for polar plots
\usetkzobj{all}
\usepackage[makeroom]{cancel} %% for strike outs
%\usepackage{mathtools} %% for pretty underbrace % Breaks Ximera
\usepackage{multicol}





\usepackage{array}
\setlength{\extrarowheight}{+.1cm}   
\newdimen\digitwidth
\settowidth\digitwidth{9}
\def\divrule#1#2{
\noalign{\moveright#1\digitwidth
\vbox{\hrule width#2\digitwidth}}}





\newcommand{\RR}{\mathbb R}
\newcommand{\R}{\mathbb R}
\newcommand{\N}{\mathbb N}
\newcommand{\Z}{\mathbb Z}

%\renewcommand{\d}{\,d\!}
\renewcommand{\d}{\mathop{}\!d}
\newcommand{\dd}[2][]{\frac{\d #1}{\d #2}}
\newcommand{\pp}[2][]{\frac{\partial #1}{\partial #2}}
\renewcommand{\l}{\ell}
\newcommand{\ddx}{\frac{d}{\d x}}

\newcommand{\zeroOverZero}{\ensuremath{\boldsymbol{\tfrac{0}{0}}}}
\newcommand{\inftyOverInfty}{\ensuremath{\boldsymbol{\tfrac{\infty}{\infty}}}}
\newcommand{\zeroOverInfty}{\ensuremath{\boldsymbol{\tfrac{0}{\infty}}}}
\newcommand{\zeroTimesInfty}{\ensuremath{\small\boldsymbol{0\cdot \infty}}}
\newcommand{\inftyMinusInfty}{\ensuremath{\small\boldsymbol{\infty - \infty}}}
\newcommand{\oneToInfty}{\ensuremath{\boldsymbol{1^\infty}}}
\newcommand{\zeroToZero}{\ensuremath{\boldsymbol{0^0}}}
\newcommand{\inftyToZero}{\ensuremath{\boldsymbol{\infty^0}}}



\newcommand{\numOverZero}{\ensuremath{\boldsymbol{\tfrac{\#}{0}}}}
\newcommand{\dfn}{\textbf}
%\newcommand{\unit}{\,\mathrm}
\newcommand{\unit}{\mathop{}\!\mathrm}
\newcommand{\eval}[1]{\bigg[ #1 \bigg]}
\newcommand{\seq}[1]{\left( #1 \right)}
\renewcommand{\epsilon}{\varepsilon}
\renewcommand{\iff}{\Leftrightarrow}

\DeclareMathOperator{\arccot}{arccot}
\DeclareMathOperator{\arcsec}{arcsec}
\DeclareMathOperator{\arccsc}{arccsc}
\DeclareMathOperator{\si}{Si}
\DeclareMathOperator{\proj}{proj}
\DeclareMathOperator{\scal}{scal}


\newcommand{\tightoverset}[2]{% for arrow vec
  \mathop{#2}\limits^{\vbox to -.5ex{\kern-0.75ex\hbox{$#1$}\vss}}}
\newcommand{\arrowvec}[1]{\tightoverset{\scriptstyle\rightharpoonup}{#1}}
\renewcommand{\vec}{\mathbf}
\newcommand{\veci}{\vec{i}}
\newcommand{\vecj}{\vec{j}}
\newcommand{\veck}{\vec{k}}
\newcommand{\vecl}{\boldsymbol{\l}}

\newcommand{\dotp}{\bullet}
\newcommand{\cross}{\boldsymbol\times}
\newcommand{\grad}{\boldsymbol\nabla}
\newcommand{\divergence}{\grad\dotp}
\newcommand{\curl}{\grad\cross}
%\DeclareMathOperator{\divergence}{divergence}
%\DeclareMathOperator{\curl}[1]{\grad\cross #1}


\colorlet{textColor}{black} 
\colorlet{background}{white}
\colorlet{penColor}{blue!50!black} % Color of a curve in a plot
\colorlet{penColor2}{red!50!black}% Color of a curve in a plot
\colorlet{penColor3}{red!50!blue} % Color of a curve in a plot
\colorlet{penColor4}{green!50!black} % Color of a curve in a plot
\colorlet{penColor5}{orange!80!black} % Color of a curve in a plot
\colorlet{fill1}{penColor!20} % Color of fill in a plot
\colorlet{fill2}{penColor2!20} % Color of fill in a plot
\colorlet{fillp}{fill1} % Color of positive area
\colorlet{filln}{penColor2!20} % Color of negative area
\colorlet{fill3}{penColor3!20} % Fill
\colorlet{fill4}{penColor4!20} % Fill
\colorlet{fill5}{penColor5!20} % Fill
\colorlet{gridColor}{gray!50} % Color of grid in a plot

\newcommand{\surfaceColor}{violet}
\newcommand{\surfaceColorTwo}{redyellow}
\newcommand{\sliceColor}{greenyellow}




\pgfmathdeclarefunction{gauss}{2}{% gives gaussian
  \pgfmathparse{1/(#2*sqrt(2*pi))*exp(-((x-#1)^2)/(2*#2^2))}%
}


%%%%%%%%%%%%%
%% Vectors
%%%%%%%%%%%%%

%% Simple horiz vectors
\renewcommand{\vector}[1]{\left\langle #1\right\rangle}


%% %% Complex Horiz Vectors with angle brackets
%% \makeatletter
%% \renewcommand{\vector}[2][ , ]{\left\langle%
%%   \def\nextitem{\def\nextitem{#1}}%
%%   \@for \el:=#2\do{\nextitem\el}\right\rangle%
%% }
%% \makeatother

%% %% Vertical Vectors
%% \def\vector#1{\begin{bmatrix}\vecListA#1,,\end{bmatrix}}
%% \def\vecListA#1,{\if,#1,\else #1\cr \expandafter \vecListA \fi}

%%%%%%%%%%%%%
%% End of vectors
%%%%%%%%%%%%%

%\newcommand{\fullwidth}{}
%\newcommand{\normalwidth}{}



%% makes a snazzy t-chart for evaluating functions
%\newenvironment{tchart}{\rowcolors{2}{}{background!90!textColor}\array}{\endarray}

%%This is to help with formatting on future title pages.
\newenvironment{sectionOutcomes}{}{} 



%% Flowchart stuff
%\tikzstyle{startstop} = [rectangle, rounded corners, minimum width=3cm, minimum height=1cm,text centered, draw=black]
%\tikzstyle{question} = [rectangle, minimum width=3cm, minimum height=1cm, text centered, draw=black]
%\tikzstyle{decision} = [trapezium, trapezium left angle=70, trapezium right angle=110, minimum width=3cm, minimum height=1cm, text centered, draw=black]
%\tikzstyle{question} = [rectangle, rounded corners, minimum width=3cm, minimum height=1cm,text centered, draw=black]
%\tikzstyle{process} = [rectangle, minimum width=3cm, minimum height=1cm, text centered, draw=black]
%\tikzstyle{decision} = [trapezium, trapezium left angle=70, trapezium right angle=110, minimum width=3cm, minimum height=1cm, text centered, draw=black]

\usepackage{tikz-cd}

\outcome{Understand Stokes' Theorem and how it can be used.}

\title[Dig-In:]{Stokes' Theorem}

\begin{document}
\begin{abstract}
Stokes' Theorem provides a computational tool for relating volumes and line integrals.
\end{abstract}
\maketitle

Recall the second Fundamental Theorem of Calculus:

\begin{theorem}[Second Fundamental Theorem of Calculus]\index{Second Fundamental Theorem of Calculus}
  Let $f$ be continuous on $[a,b]$. If $F$ is \textbf{any}
  antiderivative of $f$, then
  \[
  \int_a^b f(x)\d x = F(b)-F(a).
  \]
\end{theorem}

This theorem in some sense says that we can compute the area over a region by evaluating another function over its end points or boundary. Stokes Theorem says the same thing. Further, Stokes theorem is a generalization of Green's Theorem.

%%% This is from Dan's 2015 Notes on December 2nd

\begin{theorem}[Stokes' Theorem]\index{Stokes' Theorem}
  Let $S$ be an oriented smooth surface that is bounded by a simple, closed, smooth boundary curve $C$ with positive orientation. If $\overrightarrow{F}$ is a vector field, then
  \[
  \oint_C \overrightarrow{F} \cdot \d \overrightarrow{r} = \iint_S (\text{curl }\overrightarrow{F}) \cdot \overrightarrow{n} \d \sigma.
  \]
\end{theorem}

\begin{example}
Let $\overrightarrow{F} = < xz, xy, 3xz>$. and $C$ be the boundary of the plane $2x+y+z=2$ in the first octant.

\begin{center}
{\bf Insert picture of region here}
\end{center}


We'll use Stokes' Theorem to compute the circulation.

\begin{explanation}
First we need curl $\overrightarrow{F}$:

    \begin{align*}
      \text{curl} \overrightarrow{F} &= \nabla \times \overrightarrow{F} \\
      &= <\frac{\partial}{\partial x}, \frac{\partial}{\partial y}, \frac{\partial}{\partial z}> \times <xz, xy, 3xz> \\
      &= <\answer[given]{0}, \answer[given]{-(3z-x)},\answer[given]{y}>.
    \end{align*}

We'll project $S$ onto the $(x,y)$-plane (floor), so we want only $x$ and $y$ in $nabla\times \overrightarrow{F}$.

Therefore we will project $S$ in the $\overrightarrow{p}= <\answer[given]{0}, \answer[given]{0}, \answer[given]{1}>$ unit direction.

Given the curve $C$ being the boundary of $2x + y + z = 2$ in the first octant, we can find $ \nabla \times \overrightarrow{F}$ in terms of $x$ and $y$:

\[
\nabla \times \overrightarrow{F} = <\answer[given]{0}, \answer[given]{x - -6+6x+3y},\answer[given]{y}>.
\]

Since $S$ is given implicitly, we use $f = \answer{2x  +y + z - 2}$ to find $\overrightarrow{n} \d \sigma$:

    \begin{align*}
      \overrightarrow{n} \d \sigma &= \nabla \times \overrightarrow{F} \\
      &= \underbrace{\frac{\nabla f}{|\nabla f|}}_{{\color{red} \overrightarrow{n}}} \underbrace{\frac{|\nabla f|}{|\nabla f \cdot \overrightarrow{p}|} \d x \d y}_{{\color{red} \d \sigma}}\\
      &= <\answer[given]{2}, \answer[given]{1},\answer[given]{1}> \d x \d y.
    \end{align*}

{\bf Insert projection of Region onto the xy-plane}

The equation of the line bounding the projection of $S$ onto the $xy$-plane is $y = \answer[given]{2 - 2x}$, and in terms of $y$ is $x = \answer[given]{-0.5y + 1}$.

Using Stokes Theorem, we see the the circulation of $\overrightarrow{F}$ is 

    \begin{align*}
      \oint_C \overrightarrow{F} \cdot \d \overrightarrow{r} &= \iint_S (\nabla \times \overrightarrow{F}) \cdot \overrightarrow{n} \d \sigma \\
      &= \int_{\answer[given]{0}}^{\answer[given]{2}} \int_{\answer[given]{0}}^{\answer[given]{-0.5y + 1}} <0, -6 + 7x + 3y, y> \cdot <2,1,1> \d x \d y \\
      &=\answer[given]{-1}
    \end{align*}
\end{explanation}
\end{example}


We this leads to a cute fact:
\begin{fact}



\[
    \begin{tikzpicture}[every node/.style={midway}]
        \matrix[column sep={14em,between origins}, row sep={2em}] at (0,0) {
            \node(A) {$\overrightarrow{F}$ is conservative on $D$}; & \node(B) {$\overrightarrow{F} = \nabla f$ on $D$}; \\
            \node(C) {$\oint_C \overrightarrow{F}\cdot \d \overrightarrow{r} = 0$ for all $C$ in $D$}; & \node(D) {$\nabla \times \overrightarrow{F} = \overrightarrow{0}$ on $D$}; \\
        };

        \draw[<-] (C) -- (D) node[anchor=east]{};
        \draw[<->] (A) -- (B) node[anchor=south]{};
        \draw[->] (B) -- (D) node[anchor=west]{};
        \draw[<->] (A) -- (C) node[anchor=north]{};

    \end{tikzpicture}
\]

\begin{explanation}
    \begin{align*}
      \text{curl} &= \nabla \times \overrightarrow{F} \\
      &= <\frac{\partial}{\partial x}, \frac{\partial}{\partial y}, \frac{\partial}{\partial z}> \times <M, N, P> \\
      &= <P_y - N_z, -(P_x - M_z), N_x - M_y>.\\
      &= \overrightarrow{0} (\text{by the component test}).
    \end{align*}
\end{explanation}
\end{fact}


%%% Continue with Dan's Notes.



\end{document}
