\documentclass{ximera}

\input{../preamble.tex}
\usepackage{tikz-cd}

\outcome{Understand Stokes' Theorem and how it can be used.}

\title[Dig-In:]{Stokes' Theorem}

\begin{document}
\begin{abstract}
Stokes' Theorem provides a computational tool for relating volumes and line integrals.
\end{abstract}
\maketitle

Recall the second Fundamental Theorem of Calculus:

\begin{theorem}[Second Fundamental Theorem of Calculus]\index{Second Fundamental Theorem of Calculus}
  Let $f$ be continuous on $[a,b]$. If $F$ is \textbf{any}
  antiderivative of $f$, then
  \[
  \int_a^b f(x)\d x = F(b)-F(a).
  \]
\end{theorem}

This theorem in some sense says that we can compute the area over a region by evaluating another function over its end points or boundary. Stokes Theorem says the same thing. Further, Stokes theorem is a generalization of Green's Theorem.

%%% This is from Dan's 2015 Notes on December 2nd


\begin{theorem}[Stokes' Theorem]\index{Stokes' Theorem}
  Let $S$ be an oriented smooth surface that is bounded by a simple, closed, smooth boundary curve $C$ with positive orientation. If $\overrightarrow{F}$ is a vector field, then
  \[
  \oint_C \overrightarrow{F} \cdot \d \overrightarrow{r} = \iint_S (\text{curl }\overrightarrow{F}) \cdot \overrightarrow{n} \d \sigma.
  \]
\end{theorem}

\begin{example}
Let $\overrightarrow{F} = < xz, xy, 3xz>$. and $C$ be the boundary of the plane $2x+y+z=2$ in the first octant.

\begin{center}
{\bf Insert picture of region here}
\end{center}


We'll use Stokes' Theorem to compute the circulation.

\begin{explanation}
First we need curl $\overrightarrow{F}$:

    \begin{align*}
      \text{curl} \overrightarrow{F} &= \nabla \times \overrightarrow{F} \\
      &= <\frac{\partial}{\partial x}, \frac{\partial}{\partial y}, \frac{\partial}{\partial z}> \times <xz, xy, 3xz> \\
      &= <\answer[given]{0}, \answer[given]{-(3z-x)},\answer[given]{y}>.
    \end{align*}

We'll project $S$ onto the $(x,y)$-plane (floor), so we want only $x$ and $y$ in $nabla\times \overrightarrow{F}$.

Therefore we will project $S$ in the $\overrightarrow{p}= <\answer[given]{0}, \answer[given]{0}, \answer[given]{1}>$ unit direction.

Given the curve $C$ being the boundary of $2x + y + z = 2$ in the first octant, we can find $ \nabla \times \overrightarrow{F}$ in terms of $x$ and $y$:

\[
\nabla \times \overrightarrow{F} = <\answer[given]{0}, \answer[given]{x - -6+6x+3y},\answer[given]{y}>.
\]

Since $S$ is given implicitly, we use $f = \answer{2x  +y + z - 2}$ to find $\overrightarrow{n} \d \sigma$:

    \begin{align*}
      \overrightarrow{n} \d \sigma &= \nabla \times \overrightarrow{F} \\
      &= \underbrace{\frac{\nabla f}{|\nabla f|}}_{{\color{red} \overrightarrow{n}}} \underbrace{\frac{|\nabla f|}{|\nabla f \cdot \overrightarrow{p}|} \d x \d y}_{{\color{red} \d \sigma}}\\
      &= <\answer[given]{2}, \answer[given]{1},\answer[given]{1}> \d x \d y.
    \end{align*}

{\bf Insert projection of Region onto the xy-plane}

The equation of the line bounding the projection of $S$ onto the $xy$-plane is $y = \answer[given]{2 - 2x}$, and in terms of $y$ is $x = \answer[given]{-0.5y + 1}$.

Using Stokes Theorem, we see the the circulation of $\overrightarrow{F}$ is 

    \begin{align*}
      \oint_C \overrightarrow{F} \cdot \d \overrightarrow{r} &= \iint_S (\nabla \times \overrightarrow{F}) \cdot \overrightarrow{n} \d \sigma \\
      &= \int_{\answer[given]{0}}^{\answer[given]{2}} \int_{\answer[given]{0}}^{\answer[given]{-0.5y + 1}} <0, -6 + 7x + 3y, y> \cdot <2,1,1> \d x \d y \\
      &=\answer[given]{-1}
    \end{align*}
\end{explanation}
\end{example}


We this leads to a cute fact:
\begin{fact}

\[
    \begin{tikzpicture}[every node/.style={midway}]
        \matrix[column sep={14em,between origins}, row sep={2em}] at (0,0) {
            \node(A) {$\overrightarrow{F}$ is conservative on $D$}; & \node(B) {$\overrightarrow{F} = \nabla f$ on $D$}; \\
            \node(C) {$\oint_C \overrightarrow{F}\cdot \d \overrightarrow{r} = 0$ for all $C$ in $D$}; & \node(D) {$\nabla \times \overrightarrow{F} = \overrightarrow{0}$ on $D$}; \\
        };

        \draw[<-] (C) -- (D) node[anchor=east]{};
        \draw[<->] (A) -- (B) node[anchor=south]{};
        \draw[->] (B) -- (D) node[anchor=west]{};
        \draw[<->] (A) -- (C) node[anchor=north]{};

    \end{tikzpicture}
\]

\begin{explanation}
    \begin{align*}
      \text{curl} &= \nabla \times \overrightarrow{F} \\
      &= <\frac{\partial}{\partial x}, \frac{\partial}{\partial y}, \frac{\partial}{\partial z}> \times <M, N, P> \\
      &= <P_y - N_z, -(P_x - M_z), N_x - M_y>.\\
      &= \overrightarrow{0} (\text{by the component test}).
    \end{align*}
\end{explanation}
\end{fact}






\end{document}
