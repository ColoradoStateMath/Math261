\documentclass{ximera}

%\usepackage{todonotes}

\newcommand{\todo}{}

\usepackage{tkz-euclide}
\tikzset{>=stealth} %% cool arrow head
\tikzset{shorten <>/.style={ shorten >=#1, shorten <=#1 } } %% allows shorter vectors

\usetikzlibrary{backgrounds} %% for boxes around graphs
\usetikzlibrary{shapes,positioning}  %% Clouds and stars
\usetikzlibrary{matrix} %% for matrix
\usepgfplotslibrary{polar} %% for polar plots
\usetkzobj{all}
\usepackage[makeroom]{cancel} %% for strike outs
%\usepackage{mathtools} %% for pretty underbrace % Breaks Ximera
\usepackage{multicol}





\usepackage{array}
\setlength{\extrarowheight}{+.1cm}   
\newdimen\digitwidth
\settowidth\digitwidth{9}
\def\divrule#1#2{
\noalign{\moveright#1\digitwidth
\vbox{\hrule width#2\digitwidth}}}





\newcommand{\RR}{\mathbb R}
\newcommand{\R}{\mathbb R}
\newcommand{\N}{\mathbb N}
\newcommand{\Z}{\mathbb Z}

%\renewcommand{\d}{\,d\!}
\renewcommand{\d}{\mathop{}\!d}
\newcommand{\dd}[2][]{\frac{\d #1}{\d #2}}
\newcommand{\pp}[2][]{\frac{\partial #1}{\partial #2}}
\renewcommand{\l}{\ell}
\newcommand{\ddx}{\frac{d}{\d x}}

\newcommand{\zeroOverZero}{\ensuremath{\boldsymbol{\tfrac{0}{0}}}}
\newcommand{\inftyOverInfty}{\ensuremath{\boldsymbol{\tfrac{\infty}{\infty}}}}
\newcommand{\zeroOverInfty}{\ensuremath{\boldsymbol{\tfrac{0}{\infty}}}}
\newcommand{\zeroTimesInfty}{\ensuremath{\small\boldsymbol{0\cdot \infty}}}
\newcommand{\inftyMinusInfty}{\ensuremath{\small\boldsymbol{\infty - \infty}}}
\newcommand{\oneToInfty}{\ensuremath{\boldsymbol{1^\infty}}}
\newcommand{\zeroToZero}{\ensuremath{\boldsymbol{0^0}}}
\newcommand{\inftyToZero}{\ensuremath{\boldsymbol{\infty^0}}}



\newcommand{\numOverZero}{\ensuremath{\boldsymbol{\tfrac{\#}{0}}}}
\newcommand{\dfn}{\textbf}
%\newcommand{\unit}{\,\mathrm}
\newcommand{\unit}{\mathop{}\!\mathrm}
\newcommand{\eval}[1]{\bigg[ #1 \bigg]}
\newcommand{\seq}[1]{\left( #1 \right)}
\renewcommand{\epsilon}{\varepsilon}
\renewcommand{\iff}{\Leftrightarrow}

\DeclareMathOperator{\arccot}{arccot}
\DeclareMathOperator{\arcsec}{arcsec}
\DeclareMathOperator{\arccsc}{arccsc}
\DeclareMathOperator{\si}{Si}
\DeclareMathOperator{\proj}{proj}
\DeclareMathOperator{\scal}{scal}


\newcommand{\tightoverset}[2]{% for arrow vec
  \mathop{#2}\limits^{\vbox to -.5ex{\kern-0.75ex\hbox{$#1$}\vss}}}
\newcommand{\arrowvec}[1]{\tightoverset{\scriptstyle\rightharpoonup}{#1}}
\renewcommand{\vec}{\mathbf}
\newcommand{\veci}{\vec{i}}
\newcommand{\vecj}{\vec{j}}
\newcommand{\veck}{\vec{k}}
\newcommand{\vecl}{\boldsymbol{\l}}

\newcommand{\dotp}{\bullet}
\newcommand{\cross}{\boldsymbol\times}
\newcommand{\grad}{\boldsymbol\nabla}
\newcommand{\divergence}{\grad\dotp}
\newcommand{\curl}{\grad\cross}
%\DeclareMathOperator{\divergence}{divergence}
%\DeclareMathOperator{\curl}[1]{\grad\cross #1}


\colorlet{textColor}{black} 
\colorlet{background}{white}
\colorlet{penColor}{blue!50!black} % Color of a curve in a plot
\colorlet{penColor2}{red!50!black}% Color of a curve in a plot
\colorlet{penColor3}{red!50!blue} % Color of a curve in a plot
\colorlet{penColor4}{green!50!black} % Color of a curve in a plot
\colorlet{penColor5}{orange!80!black} % Color of a curve in a plot
\colorlet{fill1}{penColor!20} % Color of fill in a plot
\colorlet{fill2}{penColor2!20} % Color of fill in a plot
\colorlet{fillp}{fill1} % Color of positive area
\colorlet{filln}{penColor2!20} % Color of negative area
\colorlet{fill3}{penColor3!20} % Fill
\colorlet{fill4}{penColor4!20} % Fill
\colorlet{fill5}{penColor5!20} % Fill
\colorlet{gridColor}{gray!50} % Color of grid in a plot

\newcommand{\surfaceColor}{violet}
\newcommand{\surfaceColorTwo}{redyellow}
\newcommand{\sliceColor}{greenyellow}




\pgfmathdeclarefunction{gauss}{2}{% gives gaussian
  \pgfmathparse{1/(#2*sqrt(2*pi))*exp(-((x-#1)^2)/(2*#2^2))}%
}


%%%%%%%%%%%%%
%% Vectors
%%%%%%%%%%%%%

%% Simple horiz vectors
\renewcommand{\vector}[1]{\left\langle #1\right\rangle}


%% %% Complex Horiz Vectors with angle brackets
%% \makeatletter
%% \renewcommand{\vector}[2][ , ]{\left\langle%
%%   \def\nextitem{\def\nextitem{#1}}%
%%   \@for \el:=#2\do{\nextitem\el}\right\rangle%
%% }
%% \makeatother

%% %% Vertical Vectors
%% \def\vector#1{\begin{bmatrix}\vecListA#1,,\end{bmatrix}}
%% \def\vecListA#1,{\if,#1,\else #1\cr \expandafter \vecListA \fi}

%%%%%%%%%%%%%
%% End of vectors
%%%%%%%%%%%%%

%\newcommand{\fullwidth}{}
%\newcommand{\normalwidth}{}



%% makes a snazzy t-chart for evaluating functions
%\newenvironment{tchart}{\rowcolors{2}{}{background!90!textColor}\array}{\endarray}

%%This is to help with formatting on future title pages.
\newenvironment{sectionOutcomes}{}{} 



%% Flowchart stuff
%\tikzstyle{startstop} = [rectangle, rounded corners, minimum width=3cm, minimum height=1cm,text centered, draw=black]
%\tikzstyle{question} = [rectangle, minimum width=3cm, minimum height=1cm, text centered, draw=black]
%\tikzstyle{decision} = [trapezium, trapezium left angle=70, trapezium right angle=110, minimum width=3cm, minimum height=1cm, text centered, draw=black]
%\tikzstyle{question} = [rectangle, rounded corners, minimum width=3cm, minimum height=1cm,text centered, draw=black]
%\tikzstyle{process} = [rectangle, minimum width=3cm, minimum height=1cm, text centered, draw=black]
%\tikzstyle{decision} = [trapezium, trapezium left angle=70, trapezium right angle=110, minimum width=3cm, minimum height=1cm, text centered, draw=black]


%%\outcome{Understand Divergence Theorem and how it can be used.}

\title[Dig-In:]{Changing Bounds of Surface Integrals}

\begin{document}
\begin{abstract}
%%% Nothing for now
\end{abstract}
\maketitle

%%% From Dan Bates' Notes Dec 07

Recall the example we did earlier.

\begin{example}
Given a vector field $\overrightarrow{F} = <x,y,z>$ and surface $S: x^2 + y^2 + z^2 = a^2$ (a sphere of radius $a$), find the flux across $S$ in both ways.

We found that $\overrightarrow{n} = \frac{\nabla f}{|\nabla f|} = <x/a, y/a, z/a>$ and $\overrightarrow{F}\cdot \overrightarrow{n} = a$.
\end{example}


Let's try this now by parameterizing $S$:
    \begin{align*}
     \overrightarrow{r}(\varphi, \theta) &= <x,y,z> \\
     &= <a\sin(\varphi)\cos(\theta), a, \sin(\varphi)\sin(\theta), a\cos(\theta)>
    \end{align*}
    
since $\rho = a$ for a sphere.

Then $\overrightarrow{n} \d \sigma = (\overrightarrow{r}_\varphi \times \overrightarrow{r}_\theta) \d \varphi \d \theta$

\begin{example}
Consider $\oint_C \underbrace{y^2}_M \d x + \underbrace{3xy}_N \d y$  {\bf Input picture of half a bunt cake here.}

Note that Green's Theorem says that following:
    \begin{align*}
     \underline{Work} &= \oint_C M \d x + N \d y \\
     &= \oint_C \overrightarrow{F}\cdot \overrightarrow{T} \d s\\
     &= \iint_R (\nabla\times \overrightarrow{F}) \cdot \overrightarrow{k} \d x \d y \\
     &= \iint_R N_x - M_y \underbrace{\d A}_{\d x \d y}
    \end{align*}
And that 
    \begin{align*}
     \underline{Flux} &= \oint_C M \d y - N \d x \\
     &= \iint_R N_x + M_y \d A
    \end{align*}

This gives us valuable information for computations:
    \begin{align*}
     \oint_C y^2 \d x + 3xy \d y &= \iint_R 3y - 2y \d A\\
     &= \iint_R y \d A \\
     &= \iint_R \answer[given]{ r \sin(\theta)} r \d r \d \theta \\ 
     &= \int_{\answer[given]{0}}^{\answer[given]{\pi}} \int_{\answer[given]{1}}^{\answer[given]{2}} \answer[given]{ r^2 \sin(\theta)} \d r \d \theta \\ 
     &= \answer[given]{14/3}
    \end{align*}

\end{example}

\begin{question}
Notice that we included $r$ with the $\d r \d \theta$. When do you include $r$ with $\d r \d \theta$?
\begin{freeResponse}

\end{freeResponse}
\begin{feedback}
Anytime you are switching from $\d A$ or another coordinate system into polar/cylindrical coordinates, almost always.

This {\it not} the case when $r$ and $\theta$ happen to be your $u$ and $v$ (parameters) from a surface integral.

This This also holds with $\rho^2\sin(\varphi)$, etc from other coordinate systems.
\end{feedback}
\end{question}

\begin{example}
Consider the portion of the cone $y^2 = x^2 + z^2$ from $1\leq y \leq 3$ {\bf Insert cone picture, highlighting the sort of trapezoidal strips}

{\bf Insert picture of the shape's projection onto the (x,z)-plane.}

Then $\overrightarrow{r}(r,\theta) = < x,y,z> =<r\cos(\theta), r, r\sin(\theta)$ where
\[
\answer[given]{1}\leq r \leq \answer[given]{3}
0\leq \theta \leq \answer[given]{2\pi}
\]

Then 
    \begin{align*}
     \iint_S G(x,y,z) \d \sigma &= \int_0^{2\pi} \int_1^3 r\sin(\theta) | \overrightarrow{r}_r \times \overrightarrow{r}_\theta | \d r \d \theta \\
     &= \int_0^{2\pi} \int_1^3 r\sin(\theta) (\answer[given]{2^{0.5} r})\d r \d \theta \\
     &= \answer[given]{0}
    \end{align*}
\end{example}

%%% Should include Dec 08 here

%%% from Dec 09 notes

\begin{question}
Given the equations of the three planes
\[
P1: x+y+z = 4
P2 : 2x+2y+2z=6
P3: x-y+z=2
\]
and the vector equations of the two lines:
\[
L1: \vec{r}_1(t) = <t, 1, 3-t>
L2: \vec{r}_2(t) = <t,t,t>
\]
Determine whether the following statements are true or false:

$P1$ and $P2$ never intersect.
\begin{multipleChoice}
\choice[correct]{True}
\choice{False}
\end{multipleChoice}

Line $L1$ is the intersection of two of the planes.
\begin{multipleChoice}
\choice[correct]{True}
\choice{False}
\end{multipleChoice}

Lines $L1$ and $L2$ intersect.
\begin{multipleChoice}
\choice{True}
\choice[correct]{False}
\end{multipleChoice}

Plan $P3$ and line $L2$ never intersect.
\begin{multipleChoice}
\choice{True}
\choice[correct]{False}
\end{multipleChoice}
\end{question}




\end{document}
