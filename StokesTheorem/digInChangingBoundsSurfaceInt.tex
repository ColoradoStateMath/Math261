\documentclass{ximera}

\input{../preamble.tex}

%%\outcome{Understand Divergence Theorem and how it can be used.}

\title[Dig-In:]{Changing Bounds of Surface Integrals}

\begin{document}
\begin{abstract}
%%% Nothing for now
\end{abstract}
\maketitle

%%% From Dan Bates' Notes Dec 07

Recall the example we did earlier.

\begin{example}
Given a vector field $\overrightarrow{F} = <x,y,z>$ and surface $S: x^2 + y^2 + z^2 = a^2$ (a sphere of radius $a$), find the flux across $S$ in both ways.

We found that $\overrightarrow{n} = \frac{\nabla f}{|\nabla f|} = <x/a, y/a, z/a>$ and $\overrightarrow{F}\cdot \overrightarrow{n} = a$.
\end{example}


Let's try this now by parameterizing $S$:
    \begin{align*}
     \overrightarrow{r}(\varphi, \theta) &= <x,y,z> \\
     &= <a\sin(\varphi)\cos(\theta), a, \sin(\varphi)\sin(\theta), a\cos(\theta)>
    \end{align*}
    
since $\rho = a$ for a sphere.

Then $\overrightarrow{n} \d \sigma = (\overrightarrow{r}_\varphi \times \overrightarrow{r}_\theta) \d \varphi \d \theta$

\begin{example}
Consider $\oint_C \underbrace{y^2}_M \d x + \underbrace{3xy}_N \d y$  {\bf Input picture of half a bunt cake here.}

Note that Green's Theorem says that following:
    \begin{align*}
     \underline{Work} &= \oint_C M \d x + N \d y \\
     &= \oint_C \overrightarrow{F}\cdot \overrightarrow{T} \d s\\
     &= \iint_R (\nabla\times \overrightarrow{F}) \cdot \overrightarrow{k} \d x \d y \\
     &= \iint_R N_x - M_y \underbrace{\d A}_{\d x \d y}
    \end{align*}
And that 
    \begin{align*}
     \underline{Flux} &= \oint_C M \d y - N \d x \\
     &= \iint_R N_x + M_y \d A
    \end{align*}

This gives us valuable information for computations:
    \begin{align*}
     \oint_C y^2 \d x + 3xy \d y &= \iint_R 3y - 2y \d A\\
     &= \iint_R y \d A \\
     &= \iint_R \answer[given]{ r \sin(\theta)} r \d r \d \theta \\ 
     &= \int_{\answer[given]{0}}^{\answer[given]{\pi}} \int_{\answer[given]{1}}^{\answer[given]{2}} \answer[given]{ r^2 \sin(\theta)} \d r \d \theta \\ 
     &= \answer[given]{14/3}
    \end{align*}

\end{example}

\begin{question}
Notice that we included $r$ with the $\d r \d \theta$. When do you include $r$ with $\d r \d \theta$?
\begin{freeResponse}

\end{freeResponse}
\begin{feedback}
Anytime you are switching from $\d A$ or another coordinate system into polar/cylindrical coordinates, almost always.

This {\it not} the case when $r$ and $\theta$ happen to be your $u$ and $v$ (parameters) from a surface integral.

This This also holds with $\rho^2\sin(\varphi)$, etc from other coordinate systems.
\end{feedback}
\end{question}

\begin{example}
Consider the portion of the cone $y^2 = x^2 + z^2$ from $1\leq y \leq 3$ {\bf Insert cone picture, highlighting the sort of trapezoidal strips}

{\bf Insert picture of the shape's projection onto the (x,z)-plane.}

Then $\overrightarrow{r}(r,\theta) = < x,y,z> =<r\cos(\theta), r, r\sin(\theta)$ where
\[
\answer[given]{1}\leq r \leq \answer[given]{3}
0\leq \theta \leq \answer[given]{2\pi}
\]

Then 
    \begin{align*}
     \iint_S G(x,y,z) \d \sigma &= \int_0^{2\pi} \int_1^3 r\sin(\theta) | \overrightarrow{r}_r \times \overrightarrow{r}_\theta | \d r \d \theta \\
     &= \int_0^{2\pi} \int_1^3 r\sin(\theta) (\answer[given]{2^{0.5} r})\d r \d \theta \\
     &= \answer[given]{0}
    \end{align*}
\end{example}

%%% Should include Dec 08 here

%%% from Dec 09 notes

\begin{question}
Given the equations of the three planes
\[
P1: x+y+z = 4
P2 : 2x+2y+2z=6
P3: x-y+z=2
\]
and the vector equations of the two lines:
\[
L1: \vec{r}_1(t) = <t, 1, 3-t>
L2: \vec{r}_2(t) = <t,t,t>
\]
Determine whether the following statements are true or false:

$P1$ and $P2$ never intersect.
\begin{multipleChoice}
\choice[correct]{True}
\choice{False}
\end{multipleChoice}

Line $L1$ is the intersection of two of the planes.
\begin{multipleChoice}
\choice[correct]{True}
\choice{False}
\end{multipleChoice}

Lines $L1$ and $L2$ intersect.
\begin{multipleChoice}
\choice{True}
\choice[correct]{False}
\end{multipleChoice}

Plan $P3$ and line $L2$ never intersect.
\begin{multipleChoice}
\choice{True}
\choice[correct]{False}
\end{multipleChoice}
\end{question}




\end{document}
