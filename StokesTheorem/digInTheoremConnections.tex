\documentclass{ximera}

\input{../preamble.tex}
\usepackage{tikz-cd}

%%\outcome{Understand Divergence Theorem and how it can be used.}

\title[Dig-In:]{Connections Between Stokes' Theorem, the Divergence Theorem, and Green's Theorem}

\begin{document}
\begin{abstract}
%%% Nothing for now
\end{abstract}
\maketitle

%%%% Dan's Bates' Notes Dec 04 2015
The main three theorems of these sections have a similar {\it feel}. Your intuition is correct.
Stokes' Theorem and the Divergence Theorem are just generalizations of Green's Theorem!

\begin{explanation}
(\underline{Stokes Theorem and Green's Theorem})

In 2-D, Green's Theorem says the following about Flow: 

$\oint_C \overrightarrow{F} \cdot \d \overrightarrow{r} 
= \iint_R (\nabla \times \overrightarrow{F})\cdot \overrightarrow{k} \d A$.

In 3-D, Stokes Theorem says that $\oint_C \overrightarrow{F} \cdot \d \overrightarrow{r} 
= \iint_S (\nabla \times \overrightarrow{F})\cdot \overrightarrow{n} \d \sigma$.
\end{explanation}

\begin{explanation}
(\underline{Divergence Theorem and Green's Theorem})

In 2-D, Green's Theorem says the following about Flux: 

$\oint_C \overrightarrow{F} \cdot \overrightarrow{n} \d s
= \iint_R (\nabla \cdot \overrightarrow{F})\cdot \d A$.

In 3-D, the Divergence Theorem says that $\iint_S \overrightarrow{F} \cdot \d \overrightarrow{r} 
= \iiint_D (\nabla \cdot \overrightarrow{F}) \d V$.

Coincidence?..... {\bf NO WAY}!
\end{explanation}


\begin{theorem}[Unifying Fundamental 'Theorem']
The integral of a differential operator acting on a vecor field over a region equals the sum of the field components appropriate to the operator over the boundary \underline{OR}
\[
\int_{\text{region}} (\text{some derivative stuff}) = \int_{\text{boundary}} \text{non-derivative stuff}
\]
\end{theorem}



\end{document}
